\subsection{Phân tích phương sai một nhân tố (Single-factor ANOVA)}

Dữ liệu:
\begin{itemize}
    \item Y: là một biến phụ thuộc (có tính liên tục).
    \item X: là một biến nhân tố hay biến giải thích (có tính phân loại).
\end{itemize}
Mục tiêu:
\begin{itemize}
    \item Đánh giá xem biến nhân tố X có ảnh hưởng đến biến phụ thuộc Y hay không?
    \item Nói cách khác:
    \[
        H_{0}: \mu_{1} = \mu_{2} = \mu_{3} = \dots = \mu_{n}
        \]
        \[
        H_{1}: \exists i, j \text{ sao cho } \mu_{i} \neq \mu_{j}
        \]
\end{itemize}
%mùa ảnh hưởng tới phí giao hàng
Chúng em quyết định sẽ kiểm định xem, việc các mùa (season) sẽ ảnh hưởng như thế nào đến phí giao hàng (delivery\_charges) bằng phương pháp phân tích phương sai ANOVA một nhân tố. Dưới đây là phân tích phương sai ANOVA một nhân tố cho mẫu thống kê ( hàm phân tích đã được tích hợp sẵn trong R) :

\begin{figure}[!htbp]
    \centering
    \includegraphics[width=0.4\linewidth]{graphics/5.3.1.png}
    \caption{Phân tích phương sai ANOVA xét sự ảnh hưởng của mùa (season) đến phí giao hàng(delivery\_charges)}
\end{figure}

Xem hình trên ta thấy kết quả, giá trị p-value (tức Pr(>F)) là rất rất nhỏ hơn ngưỡng ý nghĩa phổ biến (0.05). Do đó, cho thấy sự khác biệt rất có ý nghĩa thống kê, có đủ bằng chứng thống kê rất mạnh để kết luận rằng, các mùa (season) có ảnh hưởng rất mạng đến phí giao hàng (delivery\_charges). Nói cách khác, sự khác biệt về phí giao hàng giữa các mùa không có yếu tố ngẫu nhiên.

%yêu cầu giao hàng nhanh ảnh hưởng tới phí giao hàng

Chúng em quyết định sẽ kiểm định xem, việc khách hàng yêu cầu giao hàng nhanh (is\_expedited\_delivery) sẽ ảnh hưởng như thế nào đến phí giao hàng (delivery\_charges) bằng phương pháp phân tích phương sai ANOVA một nhân tố. Dưới đây là phân tích phương sai ANOVA một nhân tố cho mẫu thống kê ( hàm phân tích đã được tích hợp sẵn trong R) :
\begin{figure}[!htbp]
    \centering
    \includegraphics[width=0.4\linewidth]{graphics/5.3.2.png}
    \caption{Phân tích phương sai ANOVA xét sự ảnh hưởng của việc khách hàng yêu cầu giao hàng nhanh (is\_expedited\_delivery) đến phí giao hàng(delivery\_charges)}
\end{figure}

Xem hình trên ta thấy kết quả, giá trị p-value (tức Pr(>F)) là rất rất nhỏ hơn ngưỡng ý nghĩa phổ biến (0.05). Do đó, cho thấy sự khác biệt rất có ý nghĩa thống kê, có đủ bằng chứng thống kê rất mạnh để kết luận rằng, việc khách hàng yêu cầu giao hàng nhanh (is\_expedited\_delivery) có ảnh hưởng rất mạng đến phí giao hàng (delivery\_charges). Nói cách khác, sự khác biệt về phí giao hàng giữa các mùa không có yếu tố ngẫu nhiên.

